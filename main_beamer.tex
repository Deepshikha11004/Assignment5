\documentclass{beamer}
\usetheme{CambridgeUS}
\usepackage{listings}
\usepackage{blkarray}
\usepackage{listings}
\usepackage{subcaption}
\usepackage{url}
\usepackage{tikz}
\usepackage{tkz-euclide} % loads  TikZ and tkz-base
%\usetkzobj{all}
\usetikzlibrary{calc,math}
\usepackage{float}
\renewcommand{\vec}[1]{\mathbf{#1}}
\usepackage[export]{adjustbox}
\usepackage[utf8]{inputenc}
\usepackage{amsmath}
\usepackage{amsfonts}
\usepackage{tikz}
\usepackage{hyperref}
\usepackage{bm}
\usetikzlibrary{automata, positioning}
\providecommand{\pr}[1]{\ensuremath{\Pr\left(#1\right)}}
\providecommand{\mbf}{\mathbf}
\providecommand{\qfunc}[1]{\ensuremath{Q\left(#1\right)}}
\providecommand{\sbrak}[1]{\ensuremath{{}\left[#1\right]}}
\providecommand{\lsbrak}[1]{\ensuremath{{}\left[#1\right.}}
\providecommand{\rsbrak}[1]{\ensuremath{{}\left.#1\right]}}
\providecommand{\brak}[1]{\ensuremath{\left(#1\right)}}
\providecommand{\lbrak}[1]{\ensuremath{\left(#1\right.}}
\providecommand{\rbrak}[1]{\ensuremath{\left.#1\right)}}
\providecommand{\cbrak}[1]{\ensuremath{\left\{#1\right\}}}
\providecommand{\lcbrak}[1]{\ensuremath{\left\{#1\right.}}
\providecommand{\rcbrak}[1]{\ensuremath{\left.#1\right\}}}
\providecommand{\abs}[1]{\vert#1\vert}

\newcounter{saveenumi}
\newcommand{\seti}{\setcounter{saveenumi}{\value{enumi}}}
\newcommand{\conti}{\setcounter{enumi}{\value{saveenumi}}}
\usepackage{amsmath}
\setbeamertemplate{caption}[numbered]{}                               
                               

\title{AI1110 Assignment 5}
\author{DEEPSHIKHA-CS21BTECH11016}
\date{\today}
\logo{\large \LaTeX{}}


\begin{document}
\begin{frame}
		\titlepage
	\end{frame}

\begin{frame}{Outline}
  \tableofcontents
\end{frame}
\section{Abstract}
	\begin{frame}{Abstract}
		\begin{itemize}
			\item 	This document contains the solution to Question of Chapter 12 (Probability) in the NCERT Class 12 Textbook.
		\end{itemize}
	\end{frame}
	
	
	\section{Question}
	\begin{frame}{Question}
		\begin{block}{\textbf{Probability  ex 13.1 q11.}}
			 A fair die is rolled. Consider events E = \{1,3,5\}, F = \{2,3\} and G = \{2,3,4,5\} .
	
	Find
	\begin{enumerate}
	\item \pr{E|F} and \pr{F|E}
	\item \pr{E|G} and \pr{G|E}
	\item \pr{(E \cup F)|G} and \pr{(E \cap F)|G}
	\end{enumerate}	 
		\end{block}
	
	\end{frame}

	

	\section{Solution}
	\begin{frame}{Solution}
		Let sample space S =\{1,2,3,4,5,6\}.
		\begin{table}[ht!]
	    \centering
	    \begin{tabular}{|c|c|c|c|c|}
	    \hline
	    Event & Set\\
	    \hline\hline
	    E & \{1,3,5\}   \\
	    F &  \{2,3\}    \\
	    G & \{2,3,4,5\} \\
	    E+F & \{1,2,3,5\} \\
	    EF & \{3\} \\
	    E+G & \{1,2,3,4,5\} \\
	    EG & \{3,5\} \\
	   (E+F)G & \{2,3,5\} \\
	   (EF)G & \{3\} \\
	    \hline
	    \end{tabular}
	    \caption{Events}
	    \label{tab:Table1}
	\end{table}
	
	    
	\end{frame}
	
\begin{frame}{}
		\begin{enumerate}
			\item 
			\begin{align}
         \pr{E|F}&=\frac{\pr{EF}}{\pr{F}}\\
                 &=\frac{\frac{1}{6}}{\frac{2}{6}}\\
                 &=\frac{1}{2}\\
         \pr{F|E}&=\frac{\pr{EF}}{\pr{E}}\\
                 &=\frac{\frac{1}{6}}{\frac{3}{6}}\\
                 &=\frac{1}{3}
    \end{align}
		\end{enumerate}
	\end{frame}
	
	\begin{frame}{}
		\begin{enumerate}
		\conti
			\item 
			\begin{align}
         \pr{E|G}&=\frac{\pr{EG}}{\pr{G}}\\
                 &=\frac{\frac{2}{6}}{\frac{4}{6}}\\
                 &=\frac{1}{2}\\
    \pr{G|E}&=\frac{\pr{EG}}{\pr{E}}\\
                 &=\frac{\frac{2}{6}}{\frac{3}{6}}\\
                 &=\frac{2}{3}
    \end{align}
		\end{enumerate}
	\end{frame}
	
	\begin{frame}{}
		\begin{enumerate}
		\conti
			\item 
			\begin{align}
         \pr{E+F|G}&=\frac{\pr{(E+F)G}}{\pr{G}}\\
              &=\frac{\frac{3}{6}}{\frac{4}{6}}\\
              &=\frac{3}{4}\\
    \pr{(EF)|G}&=\frac{\pr{(EF)G}}{\pr{G}}\\
               &=\frac{\frac{1}{6}}{\frac{4}{6}}\\
               &=\frac{1}{4}
    \end{align}
		\end{enumerate}
	\end{frame}

	
	
	


\end{document}
