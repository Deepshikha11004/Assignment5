\documentclass[journal,12pt,twocolumn]{IEEEtran}

\usepackage{enumitem}
\usepackage{tfrupee}
\usepackage{amsmath}
\usepackage{amssymb}
\usepackage{gensymb}
\usepackage{graphicx}
\usepackage{txfonts}

\def\inputGnumericTable{}

\usepackage[latin1]{inputenc}                                 
\usepackage{color}                                            
\usepackage{array}                                            
\usepackage{longtable}                                        
\usepackage{calc}                                             
\usepackage{multirow}                                         
\usepackage{hhline}                                           
\usepackage{ifthen}
\usepackage{caption} 

\providecommand{\pr}[1]{\ensuremath{\Pr\left(#1\right)}}
\providecommand{\cbrak}[1]{\ensuremath{\left\{#1\right\}}}
\renewcommand{\thefigure}{\arabic{table}}
\renewcommand{\thetable}{\arabic{table}}                                     
                               

\title{Assignment 5 }
\author{Deepshikha \\ \normalsize CS21BTECH11016 \\ \vspace*{20pt} \normalsize  10 May 2022 \\ \vspace*{20pt} \Large CBSE Probability Grade 12}


\begin{document}
	% The title
	\maketitle
	
	% The question
	\textbf{Exercise 13.1 11} 
	A fair die is rolled. Consider events E = \{1,3,5\}, F = \{2,3\} and G = \{2,3,4,5\} .
	
	Find
	
	
	\begin{enumerate}[label=(\roman*)]
	\item \pr{E|F} and \pr{F|E}
	\item \pr{E|G} and \pr{G|E}
	\item \pr{(E \cup F)|G} and \pr{(E \cap F)|G}
	\end{enumerate}	 
	
	% The answer
	\textbf{Solution.}
	
	
	Let sample space S =\{1,2,3,4,5,6\}.
	
	\begin{table}[ht!]
	    \centering
	    \begin{tabular}{|c|c|c|c|c|}
	    \hline
	    Event & Set\\
	    \hline\hline
	    E & \{1,3,5\}   \\
	    F &  \{2,3\}    \\
	    G & \{2,3,4,5\} \\
	    E+F & \{1,2,3,5\} \\
	    EF & \{3\} \\
	    E+G & \{1,2,3,4,5\} \\
	    EG & \{3,5\} \\
	   (E+F)G & \{2,3,5\} \\
	   (EF)G & \{3\} \\
	    \hline
	    \end{tabular}
	    \caption{Events}
	    \label{tab:Table1}
	\end{table}
 \begin{enumerate}[label=(\roman*)]
     \item 
     \begin{align}
         \pr{E|F}&=\frac{\pr{EF}}{\pr{F}}\\
                 &=\frac{\frac{1}{6}}{\frac{2}{6}}\\
                 &=\frac{1}{2}\\
         \pr{F|E}&=\frac{\pr{EF}}{\pr{E}}\\
                 &=\frac{\frac{1}{6}}{\frac{3}{6}}\\
                 &=\frac{1}{3}
    \end{align}
    \item
    \begin{align}
    \pr{E|G}&=\frac{\pr{EG}}{\pr{G}}\\
                 &=\frac{\frac{2}{6}}{\frac{4}{6}}\\
                 &=\frac{1}{2}\\
    \pr{G|E}&=\frac{\pr{EG}}{\pr{E}}\\
                 &=\frac{\frac{2}{6}}{\frac{3}{6}}\\
                 &=\frac{2}{3}
    \end{align}
    \item
    \begin{align}
    \pr{E+F|G}&=\frac{\pr{(E+F)G}}{\pr{G}}\\
              &=\frac{\frac{3}{6}}{\frac{4}{6}}\\
              &=\frac{3}{4}\\
    \pr{(EF)|G}&=\frac{\pr{(EF)G}}{\pr{G}}\\
               &=\frac{\frac{1}{6}}{\frac{4}{6}}\\
               &=\frac{1}{4}
    \end{align}
 \end{enumerate}


	
	
	


\end{document}
